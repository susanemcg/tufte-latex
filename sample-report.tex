\documentclass[symmetric, notoc, nobib]{towcenter-book}

\hypersetup{colorlinks}% uncomment this line if you prefer colored hyperlinks (e.g., for onscreen viewing)

%\usepackage[T1]{fontenc}
\usepackage{textcomp}
\usepackage[american]{babel}
\usepackage[autostyle]{csquotes}
\usepackage{endnotes}
\usepackage[notes15,strict,backend=biber,autolang=other,%
bibencoding=latin1,notetype=endonly]{biblatex-chicago}
\usepackage{ifthen}
\usepackage{setspace}
\renewcommand*{\biburlsetup}{%
  \Urlmuskip=0mu plus 2mu\relax
  \mathchardef\UrlBreakPenalty=200\relax
  \mathchardef\UrlBigBreakPenalty=100\relax
  \mathchardef\UrlEmergencyPenalty=9000\relax
  \appto\UrlSpecials{%
    \do\0{\mathchar`\0\penalty\UrlEmergencyPenalty}%
    \do\1{\mathchar`\1\penalty\UrlEmergencyPenalty}%
    \do\2{\mathchar`\2\penalty\UrlEmergencyPenalty}%
    \do\3{\mathchar`\3\penalty\UrlEmergencyPenalty}%
    \do\4{\mathchar`\4\penalty\UrlEmergencyPenalty}%
    \do\5{\mathchar`\5\penalty\UrlEmergencyPenalty}%
    \do\6{\mathchar`\6\penalty\UrlEmergencyPenalty}%
    \do\7{\mathchar`\7\penalty\UrlEmergencyPenalty}%
    \do\8{\mathchar`\8\penalty\UrlEmergencyPenalty}%
    \do\9{\mathchar`\9\penalty\UrlEmergencyPenalty}}%
  \def\UrlBreaks{%
    \do\.\do\@\do\/\do\\\do\!\do\_\do\|\do\;\do\>\do\]\do\)\do\}%
    \do\,\do\?\do\'\do\+\do\=\do\#\do\$\do\&\do\*\do\^\do\"}%
  \def\UrlBigBreaks{\do\:\do\-}}
\usepackage{url}
\urlstyle{rm}
\appto\bibsetup{\sloppy}
%\appto\biburlsetup{\Urlmuskip=0mu plus 4mu\relax}
\setlength{\dimen\footins}{9.5in}
\renewcommand{\notesname}{Citations}
\bibliography{sample}

\geometry{a5paper,left=20mm,top=15mm,headsep=1\baselineskip,textwidth=95mm,textheight=180mm,headheight=\baselineskip,marginparwidth=20mm, marginparsep=5mm}

%%
% Adding setspace to tighten leading on margin notes
\usepackage{setspace}

%%
% Adding back marginnote functionality that I've stripped from tufte-common.def in the process of creating towcenter-common.def
\usepackage{marginnote}
\renewcommand{\marginnote}[1]{{\marginpar{\raggedright\footnotesize
\setstretch{1.025}%
#1}}}

%%
% Book metadata
\title{Sample Tow Center printed report}
\author{Susan E. McGregor - susanemcg}
\publisher{Tow Center for Digital Journalism}


% use fontspec package and set titles to Georgia
\usepackage{fontspec}
\newfontfamily\titlefont{Georgia}

% add color package and define appropriate blues for title page
\usepackage{color}
\definecolor{darkblue}{RGB}{17,94,170}
\definecolor{lightblue}{RGB}{188,229,245}

% framed helps with graphics
\usepackage{framed}

% suppresses names and figure numbers on graphics
\usepackage{etoolbox}
\makeatletter
\patchcmd{\@caption}
  {\noindent\csname fnum@#1\endcsname: \ignorespaces}
  {}
  {}{}
\makeatother

%%
% Just some sample text
\usepackage{lipsum}

%%
% For nicely typeset tabular material
\usepackage{booktabs}

%%
% For graphics / images
\usepackage{graphicx}
\setkeys{Gin}{width=\linewidth,totalheight=\textheight,keepaspectratio}
\graphicspath{{graphics/}}

% The fancyvrb package lets us customize the formatting of verbatim
% environments.  We use a slightly smaller font.
\usepackage{fancyvrb}
\fvset{fontsize=\normalsize}

%%
% Prints argument within hanging parentheses (i.e., parentheses that take
% up no horizontal space).  Useful in tabular environments.
\newcommand{\hangp}[1]{\makebox[0pt][r]{(}#1\makebox[0pt][l]{)}}

%%
% Prints an asterisk that takes up no horizontal space.
% Useful in tabular environments.
\newcommand{\hangstar}{\makebox[0pt][l]{*}}

%%
% Prints a trailing space in a smart way.
\usepackage{xspace}


% Prints the month name (e.g., January) and the year (e.g., 2008)
\newcommand{\monthyear}{%
  \ifcase\month\or January\or February\or March\or April\or May\or June\or
  July\or August\or September\or October\or November\or
  December\fi\space\number\year
}


% Prints an epigraph and speaker in sans serif, all-caps type.
\newcommand{\openepigraph}[2]{%
  %\sffamily\fontsize{14}{16}\selectfont
  \begin{fullwidth}
  \sffamily\large
  \begin{doublespace}
  \noindent\allcaps{#1}\\% epigraph
  \noindent\allcaps{#2}% author
  \end{doublespace}
  \end{fullwidth}
}

% Inserts a blank page
\newcommand{\blankpage}{\newpage\hbox{}\thispagestyle{empty}\newpage}

\usepackage{units}

% Typesets the font size, leading, and measure in the form of 10/12x26 pc.
\newcommand{\measure}[3]{#1/#2$\times$\unit[#3]{pc}}

% Macros for typesetting the documentation
\newcommand{\hlred}[1]{\textcolor{Maroon}{#1}}% prints in red
\newcommand{\hangleft}[1]{\makebox[0pt][r]{#1}}
\newcommand{\hairsp}{\hspace{1pt}}% hair space
\newcommand{\hquad}{\hskip0.5em\relax}% half quad space
\newcommand{\TODO}{\textcolor{red}{\bf TODO!}\xspace}
\newcommand{\ie}{\textit{i.\hairsp{}e.}\xspace}
\newcommand{\eg}{\textit{e.\hairsp{}g.}\xspace}
\newcommand{\na}{\quad--}% used in tables for N/A cells
\providecommand{\XeLaTeX}{X\lower.5ex\hbox{\kern-0.15em\reflectbox{E}}\kern-0.1em\LaTeX}
\newcommand{\tXeLaTeX}{\XeLaTeX\index{XeLaTeX@\protect\XeLaTeX}}
% \index{\texttt{\textbackslash xyz}@\hangleft{\texttt{\textbackslash}}\texttt{xyz}}
\newcommand{\towcenterbs}{\symbol{'134}}% a backslash in tt type in OT1/T1
\newcommand{\doccmdnoindex}[2][]{\texttt{\towcenterbs#2}}% command name -- adds backslash automatically (and doesn't add cmd to the index)
\newcommand{\doccmddef}[2][]{%
  \hlred{\texttt{\towcenterbs#2}}\label{cmd:#2}%
  \ifthenelse{\isempty{#1}}%
    {% add the command to the index
      \index{#2 command@\protect\hangleft{\texttt{\towcenterbs}}\texttt{#2}}% command name
    }%
    {% add the command and package to the index
      \index{#2 command@\protect\hangleft{\texttt{\towcenterbs}}\texttt{#2} (\texttt{#1} package)}% command name
      \index{#1 package@\texttt{#1} package}\index{packages!#1@\texttt{#1}}% package name
    }%
}% command name -- adds backslash automatically
\newcommand{\doccmd}[2][]{%
  \texttt{\towcenterbs#2}%
  \ifthenelse{\isempty{#1}}%
    {% add the command to the index
      \index{#2 command@\protect\hangleft{\texttt{\towcenterbs}}\texttt{#2}}% command name
    }%
    {% add the command and package to the index
      \index{#2 command@\protect\hangleft{\texttt{\towcenterbs}}\texttt{#2} (\texttt{#1} package)}% command name
      \index{#1 package@\texttt{#1} package}\index{packages!#1@\texttt{#1}}% package name
    }%
}% command name -- adds backslash automatically
\newcommand{\docopt}[1]{\ensuremath{\langle}\textrm{\textit{#1}}\ensuremath{\rangle}}% optional command argument
\newcommand{\docarg}[1]{\textrm{\textit{#1}}}% (required) command argument
\newenvironment{docspec}{\begin{quotation}\ttfamily\parskip0pt\parindent0pt\ignorespaces}{\end{quotation}}% command specification environment
\newcommand{\docenv}[1]{\texttt{#1}\index{#1 environment@\texttt{#1} environment}\index{environments!#1@\texttt{#1}}}% environment name
\newcommand{\docenvdef}[1]{\hlred{\texttt{#1}}\label{env:#1}\index{#1 environment@\texttt{#1} environment}\index{environments!#1@\texttt{#1}}}% environment name
\newcommand{\docpkg}[1]{\texttt{#1}\index{#1 package@\texttt{#1} package}\index{packages!#1@\texttt{#1}}}% package name
\newcommand{\doccls}[1]{\texttt{#1}}% document class name
\newcommand{\docclsopt}[1]{\texttt{#1}\index{#1 class option@\texttt{#1} class option}\index{class options!#1@\texttt{#1}}}% document class option name
\newcommand{\docclsoptdef}[1]{\hlred{\texttt{#1}}\label{clsopt:#1}\index{#1 class option@\texttt{#1} class option}\index{class options!#1@\texttt{#1}}}% document class option name defined
\newcommand{\docmsg}[2]{\bigskip\begin{fullwidth}\noindent\ttfamily#1\end{fullwidth}\medskip\par\noindent#2}
\newcommand{\docfilehook}[2]{\texttt{#1}\index{file hooks!#2}\index{#1@\texttt{#1}}}
\newcommand{\doccounter}[1]{\texttt{#1}\index{#1 counter@\texttt{#1} counter}}

% Generates the index
\usepackage{makeidx}
\makeindex

\begin{document}

% Front matter
\frontmatter

% r.1 blank page
\blankpage

% l.1 blank page
\blankpage

% r.3 full title page
%\maketitle
%Using minipages to mimic existing layout for title page

\begin{minipage}[t][1\textheight]{0.40\textwidth}
\vspace{0pt}
\begin{flushleft}
\small\textsf{Tow Center for Digital Journalism\\
A Tow/Knight Report}
\end{flushleft}
\vfill
\includegraphics[width=.4\textwidth]{graphics/ColumbiaLogo.pdf}
\end{minipage}%
\hfill
\begin{minipage}[t][1\textheight]{70mm}
\vspace{0pt}
\Huge\textcolor{darkblue}{\textbf{{\titlefont SAMPLE\\
TOW CENTER\\
FOR\\
DIGITAL\\
JOURNALISM\\
REPORT\\
\textcolor{lightblue}{
SUSAN\\
MCGREGOR\\
}
}
}}
\vfill
\small\textsf{Funded by the Tow Foundation\\
and the John S. and James L. Knight Foundation}
\end{minipage}%

\blankpage

% v.4 copyright page
\newpage
\null
\vfill

\textsf{\textbf{Acknowledgements}} 
\\[1.0cm]
\noindent\textit{In the 21st century, that news is transmitted in more ways than ever before – in print, on the air and on the Web, with words, images, graphics, sounds and video. But always and in all media, we insist on the highest standards of integrity and ethical behavior when we gather and deliver the news.
}\\[0.5cm]
\noindent\textit{That means we abhor inaccuracies, carelessness, bias or distortions.}\\[0.5cm]

\noindent\textit{It means we will not knowingly introduce false information into material intended for publication or broadcast; nor will we alter photo or image content. Quotations must be accurate, and precise.}\\[0.5cm]

\noindent\textit{It means we always strive to identify all the sources of our information, shielding them with anonymity only when they insist upon it and when they provide vital information – not opinion or speculation; when there is no other way to obtain that information; and when we know the source is knowledgeable and reliable.}\\[0.5cm]

\noindent\textit{It means we don't plagiarize.}\\[0.5cm]

\noindent\textit{It means we avoid behavior or activities that create a conflict of interest and compromise our ability to report the news fairly and accurately, uninfluenced by any person or action.}\\[0.5cm]

\noindent\textit{It means we don't misidentify or misrepresent ourselves to get a story. When we seek an interview, we identify ourselves as AP journalists.}\\[0.5cm]

\noindent\textit{It means we don’t pay newsmakers for interviews, to take their photographs or to film or record them.}\\[0.5cm]

\noindent\textit{It means we must be fair. Whenever we portray someone in a negative light, we must make a real effort to obtain a response from that person. When mistakes are made, they must be corrected – fully, quickly and ungrudgingly.}\\[0.5cm]

\noindent\textit{And ultimately, it means it is the responsibility of every one of us to ensure that these standards are upheld. Any time a question is raised about any aspect of our work, it should be taken seriously.}\\[0.5cm]

\noindent \small{Copyright  \the\year\ \thanklessauthor}

\noindent \small{This volume produced with the assistance of the \textsf{tufte-\LaTeX{}} class.}

\noindent\textit{\monthyear}

% r.5 contents
\tableofcontents


% r.7 dedication
\cleardoublepage
~\vfill
\begin{doublespace}
\noindent\fontsize{16}{18}\selectfont\itshape
\nohyphenation
The text in this sample book is taken from the News Values statement of the Associated Press. The full text of this very useful document can be found online here: \href{http://www.ap.org/company/news-values}{http://www.ap.org/company/news-values}. Read it.
\end{doublespace}
\vfill
\vfill


% r.9 introduction
\cleardoublepage
\chapter*{Introduction}

An asterisk following a new chapter command means that chapter will not appear in the table of contents.


%%
% Start the main matter (normal chapters)
\mainmatter

\chapter{Standards \& Practices}

\section{ANONYMOUS SOURCES:}
Transparency is critical to our credibility with the public and our subscribers. Whenever possible, we pursue information on the record. When a newsmaker insists on background or off-the-record ground rules, we must adhere to a strict set of guidelines, enforced by AP news managers.

Under AP's rules, material from anonymous sources may be used only if:

\begin{enumerate}
\item The material is information and not opinion or speculation, and is vital to the news report.
\item The information is not available except under the conditions of anonymity imposed by the source.
\item The source is reliable, and in a position to have accurate information.
\end{enumerate}\marginnote{For more on formatting lists in \LaTeX{}, see \href{https://en.wikibooks.org/wiki/LaTeX/List_Structures}{``List Structures''}.\autocite{LaTeXLists}}

Reporters who intend to use material from anonymous sources must get approval from their news manager before sending the story to the desk. The manager is responsible for vetting the material and making sure it meets AP guidelines. The manager must know the identity of the source, and is obligated, like the reporter, to keep the source's identity confidential. Only after they are assured that the source material has been vetted should editors allow it to be transmitted.

Reporters should proceed with interviews on the assumption they are on the record. If the source wants to set conditions, these should be negotiated at the start of the interview. At the end of the interview, the reporter should try once again to move some or all of the information back on the record.

Before agreeing to use anonymous source material, the reporter should ask how the source knows the information is accurate, ensuring that the source has direct knowledge. Reporters may not agree to a source's request that AP not pursue additional comment or information.

The AP routinely seeks and requires more than one source. Stories should be held while attempts are made to reach additional sources for confirmation or elaboration. In rare cases, one source will be sufficient – when material comes from an authoritative figure who provides information so detailed that there is no question of its accuracy.

We must explain in the story why the source requested anonymity. And, when it’s relevant, we must describe the source's motive for disclosing the information. If the story hinges on documents, as opposed to interviews, the reporter must describe how the documents were obtained, at least to the extent possible.

The story also must provide attribution that establishes the source's credibility; simply quoting ``a source'' is not allowed. We should be as descriptive as possible: ``according to top White House aides'' or ``a senior official in the British Foreign Office.'' The description of a source must never be altered without consulting the reporter.

We must not say that a person declined comment when he or she is already quoted anonymously. And we should not attribute information to anonymous sources when it is obvious or well known. We should just state the information as fact.

Stories that use anonymous sources must carry a reporter's byline. If a reporter other than the bylined staffer contributes anonymous material to a story, that reporter should be given credit as a contributor to the story.

And all complaints and questions about the authenticity or veracity of anonymous material – from inside or outside the AP – must be promptly brought to the news manager's attention.

Not everyone understands “off the record” or “on background” to mean the same things. Before any interview in which any degree of anonymity is expected, there should be a discussion in which the ground rules are set explicitly.

\textbf{These are the AP’s definitions:}

On the record. The information can be used with no caveats, quoting the source by name.

Off the record. The information cannot be used for publication.

Background. The information can be published but only under conditions negotiated with the source. Generally, the sources do not want their names published but will agree to a description of their position. AP reporters should object vigorously when a source wants to brief a group of reporters on background and try to persuade the source to put the briefing on the record. These background briefings have become routine in many venues, especially with government officials.

Deep background. The information can be used but without attribution. The source does not want to be identified in any way, even on condition of anonymity.

In general, information obtained under any of these circumstances can be pursued with other sources to be placed on the record.

%%
% The back matter contains appendices, bibliographies, indices, glossaries, etc.



\backmatter

\theendnotes



\printindex

\end{document}

